%! Author = Drackaro
%! Date = 2/25/2024

% Preamble
\documentclass[11pt]{article}

%Packages
\usepackage{amsmath}
\usepackage{graphicx}
\usepackage{amssymb}

\author{
    Javier Rodríguez Sanchez C-411 \\ 
    María de Lourdes Choy Fernández C-412 \\ 
    Luis Alejandro Rodríguez Otero C-411
    }
\title{Proyecto de Simulación de eventos discretos \\ Simulación de Reparaciones}

% Document
\begin{document}
 
    \maketitle
    \newpage

    \tableofcontents
    \newpage

    \section{Introducción}

    \subsection{Breve introducción del proyecto}
    El proyecto consiste en analizar un problema de simulación de eventos discretos, implementar 
    dicha simulación y analizar los resultados obtenidos por esta. El problema concreto consiste 
    en un sistema que necesita n máquinas para funcionar y tiene m máquinas de respuesto, cada vez 
    que una máquina se rompe es susutituida por una de las de respuesto y se envía a reparar, el 
    tiempo de reparación y de duración de las máquinas siguen una distribución de variable aleatoria 
    específica, el sistema falla cuando se rompe una máquina y no se cuenta con un reemplazo para esta.


    \subsection{Objetivos y metas}
    El objetivo del proyecto es analizar distintos casos en los que llevar a cabo esta simulación, 
    variar las distribuciones con que se rompen y se arreglan las máquinas, la cantidad de máquinas 
    disponibles, asi como analizar la posibilidad de implementar más de un modelo que se ajuste al 
    problema a simular. Principalmente el proyecto consiste en encontrar el o los escenarios en que 
    el tiempo en que se rompe el sistema es mayor, es decir, que sea más duradero el sistema.


    \subsection{Variables que describen el problema}
    \begin{itemize}
        \item \textbf{T} $\longrightarrow$ Tiempo de la simulación.

        \item \textbf{F} $\longrightarrow$ Distribución que sigue el tiempo de duración de las máquinas antes 
        de romperse.
        
        \item \textbf{G} $\longrightarrow$ Distribución que sigue el tiempo de reparación de las máquinas.
        
        \item \textbf{n} $\longrightarrow$ Máquinas que necesita el sistema para funcionar.
        
        \item \textbf{m} $\longrightarrow$ Máquinas que se tienen de repuesto inicialmente.
    \end{itemize}

    \newpage

    \section{Detalles de implementación}
    \textbf{Pasos seguidos para la implementación:}

    \begin{enumerate}
        \item Selección del lenguaje de programación: Python.
        
        \item Esbozo del problema a simular previo a la implementación.
        
        \item Implementación de las distribuciones de variables aleatorias a utilizar: distribuciones.py
        
        \item Implementación de 2 modelos diferentes que se ajustan al problema a analizar (Uno asumiendo 
        la pérdida de memoria y otro asumiendo lo contrario): reparaciones.py
    
        \item Implementación de funciones auxiliares para ejecutar múltiples simulaciones y hallar la media
         y varianza de los tiempos de simulación.

        \item Implementación de métodos para ver los resultados de las simulaciones.
        
        \item Realizar múltiples simulaciones y pruebas para analizar la correctitud de los algoritmos y 
        encontrar posibles bugs.
    \end{enumerate}

    \newpage

    \section{Modelo Matemático}
    \subsection{Descripción del modelo de simulación}
    Para simular el problema en cuestión implementamos 2 modelos de simulación que nos parecían en un 
    principio equivalentes a dicho problema. \vspace{5mm}
    
    \noindent \textbf{Modelo 1:} Asumimos un modelo sin pérdida de memoria, calculamos desde un principio los tiempos en que 
    se romperán todas las máquinas y guardaremos esa información en un heap, para siempre tener acceso a la proxima 
    máquina que se romperá, cuando esto ocurra generamos el tiempo que esta tardará en repararse, añadiremos 
    una de las máquinas de repuesto y se generará su tiempo de duración para añadirlo al heap. El tiempo de 
    simulación se actualiza cada vez que una máquina se rompe o una se repara, lo que ocurra primero. La simulación 
    termina cuando una máquina se rompe y no se tienen máquinas de repuesto disponibles, devolviendo el tiempo 
    final de simulación, el cual se corresponde con el tiempo de vida del sistema. \vspace{5mm}

    \noindent \textbf{Modelo 2:} Asumimos un modelo con pérdida de memoria. Tiene un funcionamiento similar al modelo 1, 
    solo que en este asumimos la pérdida de memoria, es decir, no generamos todos los tiempos de rotura de las 
    máquinas sino que generamos solo el de la proxima máquina que se va a romper, y calculamos el tiempo de rotura 
    de la siguiente solo cuando esta se rompa. No llevamos un heap sino una única variable que contiene el tiempo 
    con el que se romperá la siguiente máquina. El tiempo de simulación se actualiza cada vez que una máquina se 
    rompe o una se repara, lo que ocurra primero. La simulación termina cuando una máquina se rompe y no se tienen 
    máquinas de repuesto disponibles, devolviendo el tiempo final de simulación, el cual se corresponde con el tiempo 
    de vida del sistema. 

    \subsection{Supuestos y restricciones}
    \begin{itemize}
        \item Una vez que una máquina se rompe es sustituida por un repuesto instantáneamente, sin lapso de tiempo, 
        asi mismo, de ser posible, comienza a ser reparada de forma instantánea.

        \item El reparador (que asumimos es una persona) no tiene conocimiento previo de cuanto tiempo le tomará 
        reparar una máquina, por lo que no aplica ninguna heurística de, por ejemplo, reparar siempre la que menos 
        tiempo le tome.
        
        \item Varias distribuciones de variables aleatorias fueron tomadas en cuenta para las simulaciones. Finalmente 
        decidimos que las más ajustadas a la realidad eran la distribución exponencial para el tiempo de rotura de las 
        máquinas y la distribución normal para el tiempo en estas se reparan, por algunas consideraciones del tipo, 
        las máquinas tienen más posibilidad de romperse mientras mas avanza el tiempo, el reparador siempre tarda un 
        tiempo entre reparación y reparación y es poco probable que repare dos máquinas instantáneamente, etc. Para 
        lograr varios pares de distribuciones para simulaciones distintas lo que se hizo fue variar el parámetro que estas 
        reciben
        
        \item Implementamos ambos modelos ya que nos parecía que el problema era equivalente si tenia o no 
        pérdida de memoria, pero aun asi, quisimos dejar que las simulaciones tuvieran la última palabra.
        
        \item Asumimos todas las máquinas como iguales, es decir, tener rotas las máquinas 2 y 9 lo interpretamos como 
        tener 2 máquinas rotas, sin importar cuales.
        
        \item El tiempo de simulación se da en unidades de tiempo, dependiendo del contexto en que se quiera analizar el 
        problema este puede ser interpretado como segundos, minutos, horas, días, etc.
        
        \item Intercambiamos la sección 3 con la sección 4 como venían indicadas en la orden del problema a la hora de 
        redactar este informe pues nos parece más adecuado describir los modelos antes de analizar sus resultados
    \end{itemize}

    \newpage

    \section{Resultados y experimentos}
    \subsection{Hallazgos de la simulación}
    Como dijimos anteriormente, implementamos 2 modelos: Modelo 1 (sin pérdida de memoria) y modelo 2 (con pérdida de memoria).

    Las simulaciones del modelo 1 resultaron (la mayoria) en una media de tiempo relativamente corta, es decir, el sistema solía 
    fallar en poco tiempo. En el caso del modelo 2 las simulaciones duraban más tiempo, tenían una media más alta. Respecto a la 
    varianza, el modelo 2 resultó tener una varianza notablemente superior a la del modelo 1 en la mayoría de los casos. Aqui se 
    muestran los resultados para algunos ejemplos de ejecución:

    \begin{table}[h]
        \begin{tabular}{|l|l|l|l|l|l|l|}
            \hline \textbf{Modelo} & \textbf{F} & \textbf{G} & \textbf{n} & \textbf{m} & \textbf{Media} & \textbf{Varianza} \\ 
            \hline 1 & Exp(4) & Norm(6, $\frac{6}{10}$) & 10 & 20 & 8.7 & 51.5 \\
            \hline 2 & Exp(4) & Norm(6, $\frac{6}{10}$) & 10 & 20 & 230 & 67 \\
            \hline 1 & Exp(10) & Norm(15, $\frac{15}{10}$) & 20 & 30 & 15.7 & 52.3 \\
            \hline 2 & Exp(10) & Norm(15, $\frac{15}{10}$) & 20 & 30 & 875.3 & 215 \\
            \hline 1 & Exp(20) & Norm(30, $\frac{30}{10}$) & 10 & 10 & 22.1 & 55.4 \\
            \hline 2 & Exp(20) & Norm(30, $\frac{30}{10}$) & 10 & 10 & 554.3 & 214.7 \\
            \hline
        \end{tabular}
    \end{table}

    \noindent \textbf{nota:} Los valores de la media y varianza son aproximaciones. Para estos resultados se hicieron un total de 1000 
    simulaciones con cada conjunto de parámetros. Estos resultados se pueden comprobar ejecutando el archivo Analisis de resultados.py 
    y probar los parámetros de la tabla u otros deseados.


    \subsection{Interpretación de los resultados}
    La razón por la que las simulaciones duran poco tiempo en el modelo 1 se debe a que todos los tiempos de rotura se calculan a la 
    vez y siguen la misma distribución, por lo que son valores que oscilan alrededor de un mismo valor, esto significa que existe una 
    probabilidad no pequeña de que todas tengan tiempos de rotura similares, y por tanto se rompan a la vez, dejando desprotejido el 
    sistema. En el caso del modelo 2 ocurre lo contrario, al calcular los tiempos de rotura de una máquina a la vez es poco probable que 
    se rompan dos máquinas simultáneamente.

    Sin embargo tambien podemos observar que los resultados del modelo 2 son más impredecibles que los del 1, al tener una varianza mayor 
    los resultados independientes de cada simulación abarcaron un conjunto más amplio de valores que los valores de las simulaciones del 
    modelo 1, la razón de que esto ocurra se basa en lo mismo, los valores de las simulaciones del modelo 1 siempre son pequeños por lo 
    explicado anteriormente, abarcan un conjunto de números pequeños, pero en el modelo 2 tenemos simulaciones que duraron mucho tiempo, y 
    otras que duraron poco, debido a que la característica de generar solo un tiempo de rotura a la vez vuelve más impredecible al modelo.


    \subsection{Necesidad de realizar el análisis estadítico de la simulación}
    El análisis estadístico puede ser útil a la hora de tener una mejor interpretación de los resultados dados por una simulación. Por ejemplo
    puede ser útil para tomar decisiones informadas basadas en los datos y no en la intuición, en nuestro caso particular puede resultarle útil 
    al jefe de la empresa del sistema que estamos analizando, a raíz de los resultados de las simulaciones este puede analizar cuáles son las 
    condiciones que debe crear en su sistema para alargar el tiempo de duración de este. El análisis estadístico tambien puede servir a la hora 
    de validar el modelo de simulación y ver que tan cercano está de la situación real. Tambien puede ser útil para tener un mejor conocimiento 
    de datos dispersos, por ejemplo, lo que se explicó anteriormente de los resultados obtenidos del modelo 2, que son resultados muy variados, como refleja 
    la varianza de las simulaciones del modelo, y por esto se hace imprescindible analizar los datos de este modelo a partir de la media. Justamente 
    Varianza y Media fueron las medidas estadísticas que tuvimos en cuenta para analizar los datos de nuestra simulación.


    \subsection{Análisis de parada de la simulación}
    El criterio de parada del problema a simular es que el sistema falle, es decir, que se rompa una máquina y no haya un repuesto para esta. Después 
    de numerosas simulaciones hay algunos aspectos a tener en cuenta en ambos modelos.

    En el modelo 1 se hace difícil tener simulaciones largas, por lo explicado anteriormente. Para casi cualquier conjunto de parámetros la simulacion 
    dura mucho menos tiempo que el que es útil analizar resultados, por lo que podemos concluir que para este modelo no tiene sentido analizar su 
    criterio de parada ya que este se comporta de manera similar para casi todos los escenarios.

    En el modelo 2 la situación cambia. Primeramente se deben elegir la distribución del tiempo de reparación y la distribución del tiempo de rotura de 
    manera cuidadosa, ya que si el reparador repara las máquinas a un ritmo mayor del que estas se rompen el tiempo de simulación es potencialmente 
    infinito. Para solucionar esto siempre intentamos que la escala de la distribución exponencial (distribución con la que se rompen las máquinas) sea 
    menor que el valor de la media de la distribución normal (distribución con la que se reparan las máquinas), de esta manera las máquinas se rompen a 
    un ritmo mayor que con el que son reparadas, por lo que el sistema irremediablemente fallará en un momento u otro. Otro escenario en el que el tiempo 
    de simulación de este modelo es potencialmente infinito es cuando el valor de n (máquinas en el sistema) es notablemente menor que el valor de 
    m (máquinas de reemplazo), es decir, cuando necesito pocas máquinas para funcionar y tengo muchas de repuesto, lógicamente, es muy improbable que el 
    sistema falle incluso si el ritmo de rotura es mayor que el ritmo de reparación. En los casos análogos a estos las simulaciones pueden terminar 
    prematuramente, es decir, con un número de n muy superior al de m o con un ritmo de reparación muy inferior al de rotura, de esta forma el sistema se 
    queda sin repuestos muy rápidamente y el sistema no tarda en fallar. Por lo que de este análisis podemos concluir que necesitamos valores de n y m 
    relativamente cercanos (preferiblemente n $\geq$ m) y elegir las distribuciones F y G de manera correcta (como se explicó más arriba) para obtener tiempos 
    de simulación adecuados para analizar datos.

    \newpage

    \section{Cálculo probabilístico para la estimación de resultados}
    Sea $T$ el tiempo estimado que tendrá el sistema para dejar de funcionar.
    Luego $T=k*E(F)$ donde $E(F)$ es el tiempo estimado que le tomará romperse 
    a una máquina, mientras que $k$ es el número de máquinas que al romperse harán que
    el sistema colapse. Luego $k$ está acotado por la cantidad de reemplazos.
    $k=m+\frac{T-E(F)}{E(G)}$, donde $m$ es el número inicial de máquinas a reemplazar y
    $E(G)$ es el tiempo esperado que le tome reparar la máquina al mecánico. $T-E(F)$ es
    el tiempo estimado de la duración del proceso menos el tiempo que le tome romperse
    a la primera máquina. Luego despejando $k$: \vspace{5mm}


    $$\frac{T}{E(F)}=m+\frac{T-E(F)}{E(G)}$$

    $$T*E(G) = m*E(F)*E(G) + T*E(F) - E(F)^2$$

    $$T=(m*E(G) - E(F)) * \frac{E(F)}{E(G) - E(F)}$$  \vspace{5mm}

    \noindent Aqui asumimos que $E(G)$ es mayor que $E(F)$, ya que de lo contrario el valor esperado
    sería potencialmente infinito.

    \noindent Por ejemplo si $m=10$, $E(G)=5$ y $E(F)=1$ el tiempo esperado seria:

    \noindent $$T=(10*5 - 1) * \frac{1}{5 - 1} = 12.25$$

    \textbf{Importante destacar que esta estimación fue concebida a partir del análisis del comportamiento
    del modelo 2, por lo que las estimaciones son acertadas solo para las simulaciones hechas por este Modelo.
    Para el modelo 1 no encontramos mecanismo de estimación de resultados.}

    \newpage

    \subsection{Ejemplos de la efectividad de la estimación}
    \begin{table}[h]
        \begin{tabular}{|l|l|l|l|l|l|l|}
            \hline \textbf{Modelo} & \textbf{F} & \textbf{G} & \textbf{n} & \textbf{m} & \textbf{Media real} & \textbf{Media estimada} \\ 
            \hline 2 & Exp(4) & Norm(6, $\frac{6}{10}$) & 10 & 20 & 230 & 232 \\
            \hline 2 & Exp(10) & Norm(15, $\frac{15}{10}$) & 20 & 30 & 875.3 & 880 \\
            \hline 2 & Exp(20) & Norm(30, $\frac{30}{10}$) & 10 & 10 & 554.3 & 560 \\
            \hline 2 & Exp(2) & Norm(4, $\frac{4}{10}$) & 30 & 15 & 59.1 & 58 \\
            \hline
        \end{tabular}
    \end{table}
    \noindent \textbf{nota:} Los valores de la media y varianza son aproximaciones. Para estos resultados se hicieron un total de 1000 
    simulaciones con cada conjunto de parámetros. Estos resultados se pueden comprobar ejecutando el archivo Analisis de resultados.py 
    y probar los parámetros de la tabla u otros deseados.
\end{document}